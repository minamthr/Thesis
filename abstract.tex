\chapter*{Abstract}
\addcontentsline{toc}{chapter}{Abstract}

\singlespacing
\doublespacing

A variety of complex organic moleculues have been observed for decades in the interstellar medium.
Some of them are considered to be delivered to the primordial Earth by comets, 
and contributed to the chemical evolution leading to terrestrial life.
One example of such prebiotic species is amino acid. Glycine, the simplest amino acid, 
has been detected in comet 67P/C-G but its presence in molecular clouds is still uncertain.

In this work we analyze the ALMA archival data toward 3 star-forming regions, 
Orion Kleinmann-Low nebula (hearafter Orion-KL), IRAS 16293-2422 (IRAS 16293), and L483,
to search methylamine (CH$_3$NH$_2$), which is suggested as precursors to glycine. 

As a result of analysis, we found 8 candidate emission at the hot core region in Orion-KL.
By using the rotation diagram method, we evaluated its tentative column density 
and rotational temperature to be $5.5^{+1.6}_{-1.1} ) \times 10^{14}$ cm$^{-2}$ and $95.4^{+15.5}_{-11.7} \,\,\mathrm{K}$, respectively. 
On the other hand, CH$_3$NH$_2$ is not detected and stringent upper limit column densities
are determined in IRAS 16293 and L483.