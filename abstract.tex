\chapter*{Abstract}
\addcontentsline{toc}{chapter}{Abstract}

\singlespacing

A variety of complex organic molecules have been observed for decades in the interstellar medium.
Some of them are considered to be delivered to the primordial Earth by comets, 
and contributed to the chemical evolution leading to terrestrial life.
One example of such prebiotic species is amino acid. Glycine, the simplest amino acid, 
has been detected in comet 81P/Wild 2 and comet 67P/Churyumov-Gerasimenko but its presence in molecular clouds 
is still uncertain. Detection of glycine in molecular clouds was attempted by several radio observations, 
but none of them succeeded.

We focused on methylamine (CH$_3$NH$_2$), which is thought to be potential interstellar precursors to glycine. 
It is confirmed by the experiment that the reaction of CH$_3$NH$_2$ with CO$_2$ in water ice 
yields glycine under UV irradiation. 
In terms of exploration in the Solar system, CH$_3$NH$_2$ is reported to exist with glycine in comets. 
However, a robust detection of CH$_3$NH$_2$ has been reported only in Sgr B2(N) in the case of molecular clouds.

In this work we analyzed the ALMA archival data toward 3 star-forming regions, 
Orion Kleinmann-Low nebula (hereafter Orion-KL), IRAS 16293-2422 (IRAS 16293), and L483.

As a result of the analysis, CH$_3$NH$_2$ was not be detected in low mass star-forming regions of IRAS 16293 and L483, 
but in high mass star-forming region of Orion-KL, 6 candidate emission lines were detected.
CH$_3$NH$_2$ concentrates in Hot Core. In addition, we found that the average LSR velocity $V_{\mathrm{LSR}}$ and FWHM line width are estimated to be 4.84~$\pm$~0.22~km~s$^{-1}$ 
and 4.16~$\pm$~0.22~km~s$^{-1}$, respectively. 

By using the rotation diagram method, we evaluated its tentative column density 
and rotational temperature to be $(5.5^{+1.6}_{-1.1} ) \times 10^{14}$ cm$^{-2}$ and $95.4^{+15.5}_{-11.7} \,\,\mathrm{K}$, respectively. 

The distribution and spectrum parameters ($V_{\mathrm{LSR}}$ and FWHM line width) of CH$_3$NH$_2$ in Orion-KL are reported for the first time.
Regarding its rotational temperature and column density, we gave more strict restrictions than previous studies.

However, since contamination is not sufficiently studied, this detection is still tentative.
We need further line identification including other frequency bands to constrain the rotational temperature and the column density.

\doublespacing