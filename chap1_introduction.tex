\chapter{Introduction
  \label{chap:introduction}}

%%% page numbering %%%
\pagenumbering{arabic}

%\section{Origin of life}
%The interstellar medium (ISM), where more than 190 molecules ranging from simple
%linear molecules to complex organic molecules (hereafter COMs) were detected, show
%chemically rich environment. Astronomers usually regard the species with more than six
%atoms as COMs. Not only O-bearing species, CH$_3$OH, CH3OCH3, HCOOCH3, but also
%N-bearing species such as CH3CH2CN and CH2CHCN are known COMs. In 2016, a chiral
%molecule propylene oxide (CH3CHHCH2O) was detected towards Sgr B2 (N) molecular
%cloud in absorption (McGuire et al. 2016). This detection implies that molecules can get
%sufficient complexity, and it will accelerate surveys of other chiral molecules, like amino
%acids.
%From this point of view, many observations were conducted to search for prebiotic
%molecules in the ISM, which might turn into the Seeds of Life when delivered to a
%planetary surface. Especially, a great attention was paid to amino acids, essential building
%blocks of terrestrial life; many surveys were made unsuccessfully to search for the simplest
%amino acid, glycine (NH2CH2COOH), towards Sgr B2 and other high-mass star forming
%regions (e.g., Brown et al. 1979; Snyder et al. 1983; Combes et al. 1996). In 2003, Kuan
%et al. (2003) claimed the %rst detection of glycine, however, several follow-up observations
%concluded denied the detection (e.g., Jones et al. 2007). The difficulty of the past glycine
%surveys would be originated from potential weakness of glycine lines and low sensitivities of
%telescopes used for the surveys.

\section{Glycine and methylamine}


\section{Star forming region}
\subsection{Orion Kleinmann-Low nebula}
\subsection{IRAS 16293-2422}
\subsection{L483}

\section{Radio observation}
\subsection{Atacama Large Millimeter Array}
%\subsection{Principle of interferometry}

\section{Purpose of this work}