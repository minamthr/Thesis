\chapter{Introduction
  \label{chap:introduction}}

%%% page numbering %%%
\pagenumbering{arabic}

%\section{Origin of life}
%The interstellar medium (ISM), where more than 190 molecules ranging from simple
%linear molecules to complex organic molecules (hereafter COMs) were detected, show
%chemically rich environment. Astronomers usually regard the species with more than six
%atoms as COMs. Not only O-bearing species, CH$_3$OH, CH3OCH3, HCOOCH3, but also
%N-bearing species such as CH3CH2CN and CH2CHCN are known COMs. In 2016, a chiral
%molecule propylene oxide (CH3CHHCH2O) was detected towards Sgr B2 (N) molecular
%cloud in absorption (McGuire et al. 2016). This detection implies that molecules can get
%sufficient complexity, and it will accelerate surveys of other chiral molecules, like amino
%acids.
%From this point of view, many observations were conducted to search for prebiotic
%molecules in the ISM, which might turn into the Seeds of Life when delivered to a
%planetary surface. Especially, a great attention was paid to amino acids, essential building
%blocks of terrestrial life; many surveys were made unsuccessfully to search for the simplest
%amino acid, glycine (NH2CH2COOH), towards Sgr B2 and other high-mass star forming
%regions (e.g., Brown et al. 1979; Snyder et al. 1983; Combes et al. 1996). In 2003, Kuan
%et al. (2003) claimed the %rst detection of glycine, however, several follow-up observations
%concluded denied the detection (e.g., Jones et al. 2007). The difficulty of the past glycine
%surveys would be originated from potential weakness of glycine lines and low sensitivities of
%telescopes used for the surveys.

\section{Glycine and methylamine}
Methylamine is considered as
a precursor of the simplest amino acid glycine. Recent experimental
studies have shown several reaction pathways to forming
glycine in water containing ices starting from CH3NH2 and CO2
subjected to high energy electrons (Holtom et al. 2005) or UV
radiation (Bossa et al. 2009; Lee et al. 2009). Under similar conditions
glycine can decompose to yield methylamine and CO2
(Ehrenfreund et al. 2001). Interstellar methylamine was first detected
toward Sgr B2 at 3.5 cm (Fourikis et al. 1974) and at 3mm
(Kaifu et al. 1974). Recently, methylamine has been detected in
a spiral galaxywith a high redshift of 0.89 located in front of the
quasar PKS 1830-211 (Muller et al. 2011). It was also observed
in cometary samples of the Stardust mission (Glavin et al. 2008).


図(??)にメチルアミンの構造を示す。

\section{Star forming region}
\subsection{Orion Kleinmann-Low nebula}
今回の研究対象としているこのOrion Kleinmann-Low (KL) 天体はオリオン星雲の
中にある赤外線星雲であり、太陽からおよそ437 pc(約1425 光年)離れた位置に存在
している[21]。この領域では、太陽の30 倍の質量をもつ巨大な星が誕生している非常
に活発な領域であり、広い範囲の周波数を観測するline survey でも数多くの分子が観
測されている[5, 22]。このOrion KL は太陽から最も近い位置にある大質量星形成領域
であり、放射強度も強く、なおかつ多くの分子が存在するため、この領域に関しては
1967 年に発見されて以来、line survey を含めた多くの研究が今までにもなされている。
Orion-KL において、多くの有機分子が存在する領域としてHot coreとCompact
ridgeと呼ばれる場所が知られている。Hot core は暖かく($\sim$150 K)、コンパクト(
$<$0.05 pc)で、密度の高い(106 cm-3)領域として知られており[23]、Compact ridge も
同様に暖かく密度の高い領域であることが分かっている[24]。しかし、化学的な性質は
異なっており、Hot core では窒素を含む分子(たとえば、NH3、CH3CN など)が多く
観測される一方で、Compact ridge では酸素を含む分子(たとえば、CH3OH、CH3OCH3
など)が多く発見されている[24]。

\subsection{IRAS 16293-2422}


\subsection{L483}

\section{Radio observation}
電波による宇宙空間の観測は、Karl G. Jansky による1930 年代の初めての宇宙電波
の発見、Grote Reber による世界初の電波望遠鏡の作成以来、宇宙空間の観測方法の1
つとして確立し、また、第2次世界大戦中のレーダー技術の進展を受けて大きな進展を
遂げている。これ以降、X 線や赤外線などの波長でも観測が行われるようになり、可視
光だけではわからない数々の宇宙の不思議を解き明かしている。ただし、大気による吸
収の影響で、全波長のうち地上から観測可能なのは可視光と電波、赤外線の一部となっ
ている。電波の領域では、20 MHz 以下の周波数は電離層を通らず、また高い周波数で
は大気の酸素や水蒸気に吸収されやすい。このため、電波望遠鏡の設置場所としては、
標高が高く、水蒸気が少ない場所が適切である。
可視光や赤外線など観測する波長により、宇宙空間で異なる現象を捉える事ができる
が、電波による宇宙空間の観測では他の波長にはない特徴がある。
・可視光では見えないような低温の物体を観測できる。
生まれた後の星は可視光を放射するが、星が生まれる前の温度の低い天体は可視光を放
射せず、電波を放射している。そのため、このような天体の研究には電波による観測が
必要である。
・波長が長い
この特徴のため、星間微粒子による吸収を受けにくく、奥の方まで観測することができ
る。
・電気的に干渉技術が容易
干渉計と呼ばれる望遠鏡では複数のアンテナを離して置いて観測し、電波を干渉させる
ことにより高い分解能を得ることができるが、電波領域ではこの方法により他の波長に
よる観測よりも高い分解能を得ている。
以上のような特徴により、電波による観測は天文学で重要な位置を占めている。

\subsection{Atacama Large Millimeter Array}
%\subsection{Principle of interferometry}

\section{Purpose of this work}