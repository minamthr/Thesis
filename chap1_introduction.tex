\chapter{Introduction
  \label{chap:introduction}}

%%% page numbering %%%
\pagenumbering{arabic}

%\section{Origin of life}
%The interstellar medium (ISM), where more than 190 molecules ranging from simple
%linear molecules to complex organic molecules (hereafter COMs) were detected, show
%chemically rich environment. Astronomers usually regard the species with more than six
%atoms as COMs. Not only O-bearing species, CH$_3$OH, CH3OCH3, HCOOCH3, but also
%N-bearing species such as CH3CH2CN and CH2CHCN are known COMs. In 2016, a chiral
%molecule propylene oxide (CH3CHHCH2O) was detected towards Sgr B2 (N) molecular
%cloud in absorption (McGuire et al. 2016). This detection implies that molecules can get
%sufficient complexity, and it will accelerate surveys of other chiral molecules, like amino
%acids.
%From this point of view, many observations were conducted to search for prebiotic
%molecules in the ISM, which might turn into the Seeds of Life when delivered to a
%planetary surface. Especially, a great attention was paid to amino acids, essential building
%blocks of terrestrial life; many surveys were made unsuccessfully to search for the simplest
%amino acid, glycine (NH2CH2COOH), towards Sgr B2 and other high-mass star forming
%regions (e.g., Brown et al. 1979; Snyder et al. 1983; Combes et al. 1996). In 2003, Kuan
%et al. (2003) claimed the %rst detection of glycine, however, several follow-up observations
%concluded denied the detection (e.g., Jones et al. 2007). The difficulty of the past glycine
%surveys would be originated from potential weakness of glycine lines and low sensitivities of
%telescopes used for the surveys.

\section{Glycine and methylamine}
Methylamine (CH$_3$NH$_2$) is considered as a precursor of the simplest amino acid glycine. 
Recent experimental studies have shown several reaction pathways to forming
glycine in water containing ices starting from CH$_3$NH$_2$ and CO$_2$
subjected to high energy electrons \citep{Holtom+2005} or UV
radiation \citep{Bossa+2009, Lee+2009}. Under similar conditions
glycine can decompose to yield methylamine and CO$_2$
\citep{Ehrenfreund+2001}. Interstellar CH$_3$NH$_2$ was first detected
toward Sgr B2 at 3.5 cm \citep{Fourikis+1974} and at 3mm
\citep{Kaifu+1974}. Recently, CH$_3$NH$_2$ has been detected in
in cometary samples of the Stardust mission \citep{Glavin+2008} and comet 67P/C-G \citep{Altwegg+2016, Altwegg+2017}.


%図(??)にメチルアミンの構造を示す。

\newpage
\section{Star forming region}
\subsection{Orion Kleinmann-Low nebula}
This Orion Kleinmann-Low nebula (hereafter Orion-KL) to be studied is an infrared nebula 
in the Orion nebula, which is located approximately 388 $\pm$ 5  pc away from the sun \citep{Kounkel+2017}. 
In this region, it is a very active area where a huge star with 30 times the mass of the sun is born, and a lot of molecules are observed also in line survey observing a wide range of frequencies. 
This Orion-KL is a large mass star formation region closest to the sun and has strong radiant intensity and still many molecules exist. Since this region was discovered in 1967,  many studies including line survey have been done so far.
In Orion-KL, a place called Hot core and Compact ridge is known as a region where many organic molecules exist. Hot core is known as a high-density (10$^6$ cm$^{-3}$) area with a warm ($\sim$ 150 K), compact ($<$ 0.05 pc), and Compact ridge is also known to be a warm and dense region. 
However, the chemical properties are different, while many molecules containing nitrogen (for example, NH$_3$, CH$_3$CN, etc.) are observed in Hot core, while molecules containing oxygen (eg, CH$_3$OH, CH$_3$OCH$_3$, etc.) are observed in Compact ridge.

%今回の研究対象としているこのOrion Kleinmann-Low (KL) 天体はオリオン星雲の中にある赤外線星雲であり、太陽からおよそ437 pc(約1425 光年)離れた位置に存在している[21]。この領域では、太陽の30 倍の質量をもつ巨大な星が誕生している非常に活発な領域であり、広い範囲の周波数を観測するline survey でも数多くの分子が観測されている[5, 22]。このOrion KL は太陽から最も近い位置にある大質量星形成領域であり、放射強度も強く、なおかつ多くの分子が存在するため、この領域に関しては1967 年に発見されて以来、line survey を含めた多くの研究が今までにもなされている。
%Orion-KL において、多くの有機分子が存在する領域としてHot coreとCompact ridgeと呼ばれる場所が知られている。Hot core は暖かく($\sim$150 K)、コンパクト($<$0.05 pc)で、密度の高い(106 cm-3)領域として知られており[23]、Compact ridge も同様に暖かく密度の高い領域であることが分かっている[24]。しかし、化学的な性質は異なっており、Hot core では窒素を含む分子(たとえば、NH3、CH3CN など)が多く観測される一方で、Compact ridge では酸素を含む分子(たとえば、CH3OH、CH3OCH3など)が多く発見されている[24]。

\subsection{IRAS 16293-2422}


\subsection{L483}

\newpage
\section{Radio observation}
The observation of outer space by radio waves has been established as one of observation methods of outer space since the discovery of the first cosmic radio wave in the 1930s by Karl G. Jansky and the creation of the world's first radio telescope by Grote Reber, It has made great progress with the progress of radar technology during World War II. Since then, observations have been carried out even at wavelengths such as X-rays and infrared, unraveling the wonders of the universe that can not be known with only visible light. However, due to the influence of atmospheric absorption, visible light, radio waves and part of infrared rays are observable from the ground of all wavelengths. In the area of radio waves, frequencies below 20 MHz do not pass through the ionosphere, and at high frequencies they are easily absorbed by atmospheric oxygen and water vapor. For this reason, it is appropriate to install the radio telescope in places with high altitude and low water vapor. Different phenomena can be captured in outer space by the wavelengths observed, such as visible light and infrared light, but in observation of outer space by radio waves there is a feature not found in other wavelengths.

\begin{itemize}
\item We can observe low temperature objects that can not be seen with visible light.\\
After being born, the star radiates visible light, but low temperature objects before the star is born 
release not visible light but radio waves. Therefore, observations by radio waves is necessary.

\item Long wavelength\\
Because of this feature, it is hard to be absorbed by interstellar microparticles, and it is possible to observe up to the far side.

\item Easy electrical interference technology
In a telescope called an interferometer, it is possible to obtain a high resolution by observing by placing a plurality of antennas apart and interfering radio waves. With this method in the radio wave range, higher resolution than observation with other wavelengths is obtained.
\end{itemize}

With the above characteristics, radio observation occupies an important position in astronomy.

\subsection{Atacama Large Millimeter Array}
%\subsection{Principle of interferometry}

\section{Purpose of this work}