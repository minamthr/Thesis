\chapter{Conclusions
  \label{chap:conclusions}}

In order to ascertain whether the presence of CH$_3$NH$_2$ in the star-formation regions is universal  
for glycine survey and further to accelerate the discussion regarding the exogenous delivery of 
prebiotic species to planets and connection between the Universe and life,
we explored the existence of CH$_3$NH$_2$ in 3 star-forming regions using ALMA archival data.

As a result of the analysis, CH$_3$NH$_2$ could not be detected in 2 out of 3 star-forming regions 
(see Appendix A), but some emission lines could be detected in Orion-KL.

According to the analysis, The physical properties of CH$_3$NH$_2$ were found as follows.
\begin{itemize}
\item From the integrated intensity maps, CH$_3$NH$_2$ concentrates in Hot Core.
\item The average LSR velocity and FWHM line width are estimated to be 4.84~$\pm$~0.22~km~s$^{-1}$ 
and 4.16~$\pm$~0.22~km~s~s$^{-1}$, respectively. $V_{\mathrm{LSR}}$ are consistent with 
those of N-bearing COMs observed toward Hot core.
On the other hand, $\Delta V_{1/2}$ of CH$_3$NH$_2$ is narrower than those of other molecule in Hot core.
\item By using the rotation diagram method, we evaluated its rotational temperature and column density to be $95.4^{+15.5}_{-11.7} \,\,\mathrm{K}$,  $ (\,5.5^{+1.6}_{-1.1}\,) \times 10^{14} \,\,\mathrm{cm^{-2}}$.
\end{itemize}

The distribution and spectrum parameters ($V_{\mathrm{LSR}}$ and FWHM line width) of CH$_3$NH$_2$ in Orion-KL is reported for the first time.
Regarding its rotational temperature and column density, we gave strict restrictions over previous studies.

However, since contamination can not be sufficiently studied, this detection is still tentative.
We need further line identification including other frequency bands to constrain $T_{\mathrm{rot}}$ and $N_{\mathrm{MA}}$.



